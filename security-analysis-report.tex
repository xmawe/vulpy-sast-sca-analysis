% Security Analysis Report - Vulpy Project
\documentclass[12pt,a4paper]{article}

% Packages
\usepackage[utf8]{inputenc}
\usepackage[T1]{fontenc}
\usepackage[french]{babel}
\usepackage{geometry}
\usepackage{xcolor}
\usepackage{listings}
\usepackage{hyperref}
\usepackage{fancyhdr}
\usepackage{titlesec}
\usepackage{booktabs}
\usepackage{graphicx}
\usepackage{longtable}
\usepackage{float}
\usepackage{enumitem}

% Page geometry
\geometry{left=2.5cm, right=2.5cm, top=3cm, bottom=3cm}

% Colors
\definecolor{backcolour}{rgb}{0.95,0.95,0.92}
\definecolor{high}{rgb}{0.8,0.1,0.1}
\definecolor{medium}{rgb}{0.9,0.5,0.1}

% Code listing style
\lstdefinestyle{mystyle}{
    backgroundcolor=\color{backcolour},
    basicstyle=\ttfamily\footnotesize,
    breaklines=true,
    numbers=left,
    numbersep=5pt
}
\lstset{style=mystyle}

% Headers and footers
\pagestyle{fancy}
\fancyhf{}
\fancyhead[L]{Rapport d'Analyse de Sécurité}
\fancyhead[R]{Projet Vulpy}
\fancyfoot[C]{\thepage}

% Title formatting
\titleformat{\section}{\Large\bfseries}{\thesection}{1em}{}
\titleformat{\subsection}{\large\bfseries}{\thesubsection}{1em}{}

% Hyperlink setup
\hypersetup{
    colorlinks=true,
    linkcolor=blue,
    filecolor=magenta,      
    urlcolor=cyan,
    pdftitle={Rapport d'Analyse de Sécurité - Vulpy},
    pdfpagemode=FullScreen,
}

% Document
\begin{document}

% Title page
\begin{titlepage}
    \centering
    \vspace*{3cm}
    
    {\Huge\bfseries Rapport d'Analyse de Sécurité\\[0.5cm]}
    {\LARGE Projet Vulpy\\[2cm]}
    
    {\large Analyse SAST, SCA et DAST avec Jenkins, Bandit, Trivy et OWASP ZAP\\[3cm]}
    
    {\large Date : 11 décembre 2025\\[0.5cm]}
    {\large Réalisé par : JADA Mohamed\\[0.5cm]}
\end{titlepage}

% Table of contents
\tableofcontents
\newpage

% Executive Summary
\section{Résumé Exécutif}

Ce rapport présente une analyse complète de sécurité du projet Vulpy, une application web Flask volontairement vulnérable, en utilisant des outils d'analyse statique (SAST), d'analyse de composition logicielle (SCA), et de tests dynamiques (DAST). L'objectif est d'identifier, documenter et corriger les vulnérabilités présentes dans le code source, les dépendances et la configuration runtime de l'application.

\subsection*{Informations du Projet}
\begin{itemize}[leftmargin=*]
    \item \textbf{Lien du projet :} \url{https://github.com/xmawe/vulpy-sast-sca-analysis}
    
    \item \textbf{Rapports SAST avant corrections :} \url{https://github.com/xmawe/vulpy-sast-sca-analysis/tree/main/reports/sast/before-vul-correction}
    
    \item \textbf{Rapports SAST après corrections :} \url{https://github.com/xmawe/vulpy-sast-sca-analysis/tree/main/reports/sast/after-vul-correction}
    
    \item \textbf{Rapports DAST avant corrections :} \url{https://github.com/xmawe/vulpy-sast-sca-analysis/tree/main/reports/dast/before-val-correction}
    
    \item \textbf{Rapports DAST après corrections :} \url{https://github.com/xmawe/vulpy-sast-sca-analysis/tree/main/reports/dast/after-val-correction}
\end{itemize}

\subsection*{Résultats Clés}
\begin{itemize}[leftmargin=*]
    \item \textbf{Version bad (vulnérable) :} 22 vulnérabilités détectées
    \begin{itemize}
        \item 2 vulnérabilités HIGH (Flask debug mode)
        \item 13 vulnérabilités MEDIUM (SQL injection, hardcoded /tmp, timeout)
        \item 7 vulnérabilités LOW (hardcoded secrets, subprocess, try/except)
    \end{itemize}
    \item \textbf{Version good (après corrections) :} 11 vulnérabilités détectées initialement
    \begin{itemize}
        \item 2 vulnérabilités HIGH (Flask debug mode)
        \item 5 vulnérabilités MEDIUM (hardcoded /tmp, timeout, SQL injection)
        \item 4 vulnérabilités LOW (hardcoded secrets, try/except)
    \end{itemize}
    \item \textbf{Corrections appliquées :} Toutes les vulnérabilités HIGH et MEDIUM éliminées (100\%)
    \item \textbf{État final :} 0 vulnérabilité dans la version good corrigée
    \item 0 vulnérabilité CVE dans les dépendances Python
\end{itemize}

\newpage

% Objectives
\section{Objectifs}

\begin{enumerate}[leftmargin=*]
    \item Pipeline CI/CD avec Jenkins
    \item Analyse SAST avec Bandit
    \item Analyse SCA avec Trivy
    \item Analyse DAST avec OWASP ZAP
    \item Correction des vulnérabilités critiques et moyennes
\end{enumerate}

\subsection{Outils Utilisés}

Bandit 1.8.6 (SAST), Trivy 0.48.0 (SCA), OWASP ZAP (DAST), Jenkins (CI/CD), Docker

\newpage

% Environment Configuration
\section{Configuration de l'Environnement}

\subsection{docker-compose.yml}

\begin{lstlisting}[language=Docker, caption=Configuration Docker Compose]
services:
  jenkins:
    build:
      context: .
      dockerfile: Dockerfile.jenkins
    image: jenkins-complete:local
    container_name: jenkins
    user: root
    ports:
      - "8080:8080"
      - "50000:50000"
    volumes:
      - jenkins_home:/var/jenkins_home
      - .:/vulpy  # Code source
      - /var/run/docker.sock:/var/run/docker.sock
    environment:
      - DOCKER_HOST=unix:///var/run/docker.sock
    networks:
      - default

volumes:
  jenkins_home:

networks:
  default:
    name: vulpy-sast-sca-analysis_default
\end{lstlisting}

\subsection{Dockerfile (Application Vulpy)}

\begin{lstlisting}[language=Docker, caption=Dockerfile optimisé pour l'application]
# Image de base Python optimisée
FROM python:3.11-slim-bookworm

# Métadonnées
LABEL maintainer="security-team"
LABEL version="1.0"
LABEL description="Vulpy Security Analysis Application"

# Variables d'environnement
ENV PYTHONUNBUFFERED=1 \
    PYTHONDONTWRITEBYTECODE=1 \
    APP_HOME=/app

WORKDIR $APP_HOME

# Installation des dépendances système
RUN apt-get update && \
    apt-get install -y --no-install-recommends \
    gcc \
    && rm -rf /var/lib/apt/lists/*

# Copie et installation des dépendances Python
COPY requirements.txt .
RUN pip install --no-cache-dir --upgrade pip && \
    pip install --no-cache-dir -r requirements.txt

# Copie du code source
COPY . .

# Initialisation de la base de données
WORKDIR $APP_HOME/bad
RUN python db_init.py || echo "DB initialization completed"

# Création d'un utilisateur non-root
RUN useradd -m -u 1000 vulpyuser && \
    chown -R vulpyuser:vulpyuser $APP_HOME

USER vulpyuser

EXPOSE 5000

HEALTHCHECK --interval=30s --timeout=3s --start-period=5s --retries=3 \
    CMD python -c "import requests; requests.get('http://localhost:5000')" || exit 1

CMD ["python", "vulpy.py"]
\end{lstlisting}

\subsection{Pipeline Jenkins}

Le pipeline exécute plusieurs étapes : SAST (Bandit), SCA (Trivy sur dépendances, secrets, supply chain), build Docker, scan d'image, et DAST (OWASP ZAP).

\subsection{Dockerfile.jenkins}

\begin{lstlisting}[language=Docker, caption=Image Jenkins personnalisée avec outils de sécurité]
# Image de base Jenkins LTS
FROM jenkins/jenkins:lts-jdk17

# Passage en root pour installations
USER root

# Installation Docker CLI (version stable)
RUN apt-get update && \
    apt-get install -y --no-install-recommends \
        apt-transport-https \
        ca-certificates \
        curl \
        gnupg \
        lsb-release && \
    install -m 0755 -d /etc/apt/keyrings && \
    curl -fsSL https://download.docker.com/linux/debian/gpg \
        -o /etc/apt/keyrings/docker.asc && \
    chmod a+r /etc/apt/keyrings/docker.asc && \
    echo "deb [arch=amd64 signed-by=/etc/apt/keyrings/docker.asc] \
        https://download.docker.com/linux/debian bookworm stable" \
        > /etc/apt/sources.list.d/docker.list && \
    apt-get update && \
    apt-get install -y docker-ce-cli docker-compose-plugin && \
    apt-get clean && rm -rf /var/lib/apt/lists/*

# Installation Python 3 et outils de sécurité
RUN apt-get update && \
    apt-get install -y --no-install-recommends \
        python3 \
        python3-pip \
        python3-venv \
        git \
        wget && \
    pip3 install --break-system-packages \
        bandit==1.8.6 \
        safety \
        pylint && \
    apt-get clean && rm -rf /var/lib/apt/lists/*

# Installation Trivy (scanner de vulnérabilités)
RUN wget -qO - https://aquasecurity.github.io/trivy-repo/deb/public.key \
        | gpg --dearmor -o /usr/share/keyrings/trivy.gpg && \
    echo "deb [signed-by=/usr/share/keyrings/trivy.gpg] \
        https://aquasecurity.github.io/trivy-repo/deb generic main" \
        > /etc/apt/sources.list.d/trivy.list && \
    apt-get update && \
    apt-get install -y trivy && \
    apt-get clean && rm -rf /var/lib/apt/lists/*

# Vérification des versions installées
RUN echo "=== Versions des outils ==="  && \
    docker --version && \
    python3 --version && \
    bandit --version && \
    trivy --version && \
    echo "=== Installation terminée ==="

# Configuration des permissions
RUN usermod -aG docker jenkins

# Retour à l'utilisateur jenkins
USER jenkins

# Variables d'environnement Jenkins
ENV JAVA_OPTS="-Djenkins.install.runSetupWizard=false"
\end{lstlisting}

\newpage

\subsection{Exécution du Pipeline}

Pipeline exécuté avec succès : 14 étapes validées, générant des rapports séparés pour les versions bad (vulnérable) et good (sécurisée).

\begin{figure}[H]
    \centering
    \includegraphics[width=1\linewidth]{img-2.png}
        \caption{Exécution du Pipeline}
    \label{fig:placeholder}
\end{figure}

\subsection{Artefacts Générés}
Le pipeline génère des rapports séparés pour chaque version (bad et good) :
\begin{itemize}[leftmargin=*]
    \item 2 rapports Bandit HTML (bad + good)
    \item 8 rapports Trivy JSON (4 par version : dependencies, secrets, supply-chain, container)
    \item 6 rapports DAST OWASP ZAP (3 formats × 2 versions : HTML, XML, JSON)
    \item 2 fichiers de dépendances (all-dependencies.txt, all-deps.txt)
    \item 1 rapport requirements.txt
\end{itemize}

Total : environ 19 fichiers par exécution du pipeline.

\begin{figure}[H]
    \centering
    \includegraphics[width=1\linewidth]{img-1.png}
    \caption{Artefacts Générés}
    \label{fig:placeholder}
\end{figure}

\newpage

% SAST Analysis
\section{Analyse SAST avec Bandit}

\subsection{Analyse SAST avec Bandit}
Le SAST (Static Application Security Testing) examine le code source sans l'exécuter. Bandit analyse l'AST du code Python, détecte les patterns dangereux et attribue des niveaux de sévérité.

\subsection{Types de Vulnérabilités Détectées par Bandit}

\begin{longtable}{@{}p{2cm}p{5cm}p{6cm}@{}}
\toprule
\textbf{Test ID} & \textbf{Nom} & \textbf{Description} \\ \midrule
\endfirsthead
\toprule
\textbf{Test ID} & \textbf{Nom} & \textbf{Description} \\ \midrule
\endhead
B105 & hardcoded\_password\_string & Mot de passe codé en dur \\
B108 & hardcoded\_tmp\_directory & Utilisation non sécurisée de /tmp \\
B110 & try\_except\_pass & Exception ignorée silencieusement \\
B113 & request\_without\_timeout & Requête HTTP sans timeout \\
B201 & flask\_debug\_true & Mode debug Flask activé \\
B311 & random & Générateur aléatoire non cryptographique \\
B404 & subprocess & Import du module subprocess \\
B603 & subprocess\_without\_shell & Appel subprocess potentiellement dangereux \\
B608 & hardcoded\_sql\_expressions & Injection SQL potentielle \\ \bottomrule
\caption{Types de vulnérabilités détectées par Bandit}
\label{tab:bandit_types}
\end{longtable}

\subsection{Résultats de l'Analyse - Code bad/}

\textbf{Total : 22 vulnérabilités détectées par Bandit}

\begin{table}[H]
\centering
\small
\begin{tabular}{@{}p{1.5cm}p{3.5cm}p{2.5cm}p{2cm}p{4cm}@{}}
\toprule
\textbf{Test ID} & \textbf{Type} & \textbf{Fichier} & \textbf{Sévérité} & \textbf{Description} \\ \midrule
B201 & flask\_debug\_true & vulpy.py:55 & \textcolor{high}{HIGH} & Exécution de code arbitraire \\
B201 & flask\_debug\_true & vulpy-ssl.py:29 & \textcolor{high}{HIGH} & Werkzeug debugger exposé \\
B608 & SQL Injection & libuser.py:12,25,53 & \textcolor{medium}{MEDIUM} & Construction string SQL \\
B608 & SQL Injection & db.py:19 & \textcolor{medium}{MEDIUM} & INSERT sans paramètres \\
B608 & SQL Injection & db\_init.py:20 & \textcolor{medium}{MEDIUM} & Concaténation SQL \\
B108 & hardcoded\_tmp & libapi.py:16,20,33 & \textcolor{medium}{MEDIUM} & /tmp/ non sécurisé \\
B108 & hardcoded\_tmp & api\_post.py:6 & \textcolor{medium}{MEDIUM} & Fichier temporaire \\
B108 & hardcoded\_tmp & vulpy-ssl.py:29 & \textcolor{medium}{MEDIUM} & Certificats SSL en /tmp \\
B113 & no\_timeout & api\_list.py:10 & \textcolor{medium}{MEDIUM} & requests.get sans timeout \\
B113 & no\_timeout & api\_post.py:16,30 & \textcolor{medium}{MEDIUM} & requests.post sans timeout \\
B105 & hardcoded\_password & vulpy.py:16 & LOW & SECRET\_KEY = 'aaaaaaa' \\
B105 & hardcoded\_password & vulpy-ssl.py:13 & LOW & SECRET\_KEY codé en dur \\
B311 & weak\_random & libapi.py:14 & LOW & random.getrandbits() \\
B404 & subprocess\_import & brute.py:3 & LOW & Import subprocess \\
B603 & subprocess\_call & brute.py:21 & LOW & subprocess.run() \\
B110 & try\_except\_pass & libsession.py:21 & LOW & Exception ignorée \\ \bottomrule
\end{tabular}
\caption{Vulnérabilités réelles détectées dans le code bad/}
\label{tab:vuln_bad}
\end{table}

\subsection{Comparaison bad/ vs good/ (avant et après corrections)}

\begin{table}[H]
\centering
\begin{tabular}{@{}lccc@{}}
\toprule
\textbf{Métrique} & \textbf{bad/} & \textbf{good/ (avant)} & \textbf{good/ (après)} \\ \midrule
HIGH severity & 2 & 2 & \textcolor{green}{\textbf{0}} \\
MEDIUM severity & 13 & 5 & \textcolor{green}{\textbf{0}} \\
LOW severity & 7 & 4 & \textcolor{green}{\textbf{0}} \\
\textbf{TOTAL} & \textbf{22} & \textbf{11} & \textcolor{green}{\textbf{0}} \\
Lines of code & 495 & 567 & 567 \\ \bottomrule
\end{tabular}
\caption{Comparaison des vulnérabilités : bad/ vs good/ avant et après corrections}
\label{tab:comparison}
\end{table}

\textbf{Analyse :} Les corrections appliquées ont permis d'éliminer \textbf{100\% des vulnérabilités} détectées dans la version good. Le répertoire \texttt{bad/} conserve intentionnellement toutes ses vulnérabilités à des fins pédagogiques.

\subsection{Indicateurs Clés}

\textbf{Severity:} LOW (faible), MEDIUM (moyen), HIGH (critique)

\textbf{Confidence:} Niveau de certitude de Bandit

\textbf{CWE:} Common Weakness Enumeration - Classification standardisée des vulnérabilités

\newpage

% SCA Analysis
\section{Analyse SCA avec Trivy}

Le SCA (Software Composition Analysis) analyse les dépendances tierces pour identifier les vulnérabilités CVE connues. Trivy scanne les dépendances, images Docker, et secrets exposés.

\subsection{Résultat du Scan}
32 dépendances analysées :

\textbf{Dépendances principales :} Flask 3.1.2, requests 2.32.5, cryptography 46.0.3, PyJWT 2.10.1, pyotp 2.9.0, geoip2 5.2.0, qrcode 8.2

\textbf{Dépendances transitives :} Werkzeug 3.1.4, Jinja2 3.1.6, urllib3 2.5.0, aiohttp 3.13.2, certifi 2025.11.12, click 8.3.1

\textbf{Résultat :} \textcolor{green}{\textbf{Aucune vulnérabilité CVE}} détectée dans les dépendances Python. Toutes les bibliothèques sont à jour (versions novembre-décembre 2025).

\subsection{Vulnérabilités Détectées dans l'Image Docker}

\begin{table}[H]
\centering
\begin{tabular}{@{}llllll@{}}
\toprule
\textbf{CVE} & \textbf{Package} & \textbf{Sévérité} & \textbf{Version Affectée} & \textbf{Fix} \\ \midrule
CVE-2025-8869 & pip & \textcolor{medium}{MEDIUM} & 24.0 & 25.3 \\
DS002 & Docker & \textcolor{high}{HIGH} & N/A & Ajouter USER \\
DS029 & Docker & \textcolor{high}{HIGH} & N/A & --no-install-recommends \\
CVE-2025-7709 & sqlite & \textcolor{medium}{MEDIUM} & 3.46.1 & Mise à jour \\ \bottomrule
\end{tabular}
\caption{Vulnérabilités détectées par Trivy}
\label{tab:trivy_vulns}
\end{table}

\newpage

% DAST Analysis
\section{Analyse DAST avec OWASP ZAP}

Le DAST (Dynamic Application Security Testing) effectue des tests de sécurité sur l'application en cours d'exécution, simulant des attaques réelles pour détecter les vulnérabilités de configuration et d'implémentation qui ne peuvent pas être identifiées par l'analyse statique.

\subsection{Configuration de l'Environnement DAST}

\subsubsection{Outils Utilisés}

\begin{itemize}[leftmargin=*]
    \item \textbf{OWASP ZAP (Zed Attack Proxy)} : Scanner de sécurité open-source
    \item \textbf{Docker} : Conteneurisation des applications et de ZAP
    \item \textbf{Jenkins Pipeline} : Automatisation du scan DAST
\end{itemize}

\subsubsection{Architecture de Test}

Les applications sont déployées dans des conteneurs Docker connectés au réseau \texttt{vulpy-sast-sca-analysis\_default} :

\begin{itemize}[leftmargin=*]
    \item \textbf{Version Bad} : \texttt{http://vulpy-bad-app:5000/} (port externe 5001)
    \item \textbf{Version Good} : \texttt{http://vulpy-good-app:5000/} (port externe 5002)
    \item \textbf{ZAP Scanner} : Conteneur éphémère lancé pour chaque scan
\end{itemize}

\subsubsection{Configuration du Scan}

\begin{lstlisting}[language=bash, caption=Commande ZAP Baseline Scan]
docker run --rm -u root \
    --network vulpy-sast-sca-analysis_default \
    -v ${WORKSPACE}/reports:/zap/wrk:rw \
    zaproxy/zap-stable \
    zap-baseline.py \
        -t http://vulpy-bad-app:5000/ \
        -r zap-baseline-report-bad.html \
        -x zap-baseline-report-bad.xml \
        -J zap-baseline-report-bad.json \
        -I
\end{lstlisting}

Paramètres :
\begin{itemize}[leftmargin=*]
    \item \texttt{-u root} : Permissions d'écriture pour les rapports
    \item \texttt{-t} : URL cible à scanner
    \item \texttt{-r/-x/-J} : Génération des rapports HTML, XML et JSON
    \item \texttt{-I} : Ignore les avertissements pour ne pas bloquer le pipeline
\end{itemize}

\subsection{Résultats du Scan DAST}

\subsubsection{Version Bad (Vulnérable)}

Le scan OWASP ZAP a détecté \textbf{16 avertissements} (WARN-NEW) dans la version vulnérable :

\begin{table}[H]
\centering
\small
\begin{tabular}{@{}p{7cm}p{2cm}p{1.5cm}@{}}
\toprule
\textbf{Vulnérabilité} & \textbf{Code ZAP} & \textbf{Occurrences} \\ \midrule
Cross-Domain JavaScript Source File Inclusion & 10017 & 1 \\
\textcolor{red}{\textbf{Missing Anti-clickjacking Header}} & \textbf{10020} & \textbf{8} \\
\textcolor{red}{\textbf{X-Content-Type-Options Header Missing}} & \textbf{10021} & \textbf{12} \\
Information Disclosure - Suspicious Comments & 10027 & 2 \\
Server Leaks Version Information & 10036 & 11 \\
Content Security Policy Header Not Set & 10038 & 12 \\
Non-Storable Content & 10049 & 13 \\
\textcolor{red}{\textbf{Cookie without SameSite Attribute}} & \textbf{10054} & \textbf{1} \\
Permissions Policy Header Not Set & 10063 & 13 \\
Modern Web Application & 10109 & 1 \\
Authentication Request Identified & 10111 & 2 \\
Session Management Response Identified & 10112 & 1 \\
\textcolor{red}{\textbf{Absence of Anti-CSRF Tokens}} & \textbf{10202} & \textbf{3} \\
Sub Resource Integrity Attribute Missing & 90003 & 10 \\
Insufficient Site Isolation Against Spectre & 90004 & 12 \\
Application Error Disclosure & 90022 & 1 \\ \bottomrule
\end{tabular}
\caption{Vulnérabilités DAST détectées - Version Bad}
\end{table}

\textbf{Total :} FAIL-NEW: 0 | FAIL-INPROG: 0 | \textcolor{medium}{\textbf{WARN-NEW: 16}} | WARN-INPROG: 0 | INFO: 0 | PASS: 51

\subsubsection{Version Good (Après Corrections DAST)}

Le scan sur la version corrigée montre \textbf{~10 avertissements} après les corrections appliquées :

\begin{table}[H]
\centering
\small
\begin{tabular}{@{}p{7cm}p{2cm}p{1.5cm}@{}}
\toprule
\textbf{Vulnérabilité} & \textbf{Code ZAP} & \textbf{Statut} \\ \midrule
Cross-Domain JavaScript Source File Inclusion & 10017 & ⚠️ Présent \\
\textcolor{green}{\textbf{Missing Anti-clickjacking Header}} & \textbf{10020} & \textcolor{green}{\textbf{✓ CORRIGÉ}} \\
\textcolor{green}{\textbf{X-Content-Type-Options Header Missing}} & \textbf{10021} & \textcolor{green}{\textbf{✓ CORRIGÉ}} \\
Server Leaks Version Information & 10036 & ⚠️ Présent \\
Non-Storable Content & 10049 & ⚠️ Présent \\
Cookie without SameSite Attribute & 10054 & ⚠️ Présent \\
CSP: Failure to Define Directive with No Fallback & 10055 & ⚠️ Présent \\
Permissions Policy Header Not Set & 10063 & ⚠️ Présent \\
Authentication Request Identified & 10111 & ⚠️ Présent \\
Session Management Response Identified & 10112 & ⚠️ Présent \\
\textcolor{green}{\textbf{Absence of Anti-CSRF Tokens}} & \textbf{10202} & \textcolor{green}{\textbf{✓ CORRIGÉ}} \\
Sub Resource Integrity Attribute Missing & 90003 & ⚠️ Présent \\
Insufficient Site Isolation Against Spectre & 90004 & ⚠️ Présent \\ \bottomrule
\end{tabular}
\caption{Comparaison DAST - Version Good}
\end{table}

\subsection{Corrections DAST Appliquées}

Trois vulnérabilités critiques identifiées par le scan DAST ont été corrigées dans la version good :

\subsubsection{1. Protection Anti-CSRF (Cross-Site Request Forgery) [10202]}

\textbf{Problème :} Les formulaires de l'application ne comportaient aucune protection contre les attaques CSRF, permettant à un attaquant de forcer un utilisateur authentifié à exécuter des actions non désirées (changement de mot de passe, modification de profil, etc.).

\textbf{Impact :} CRITIQUE - Un attaquant peut créer une page malveillante qui soumet automatiquement des formulaires au nom de la victime si celle-ci est authentifiée.

\textbf{Solution Implémentée :}

\begin{enumerate}[leftmargin=*]
    \item \textbf{Installation de Flask-WTF :}
    \begin{lstlisting}[language=bash]
# Ajout dans requirements.txt
Flask-WTF
    \end{lstlisting}
    
    \item \textbf{Configuration de CSRFProtect :}
    \begin{lstlisting}[language=Python]
# vulpy/good/vulpy.py
from flask_wtf.csrf import CSRFProtect

app.config['WTF_CSRF_TIME_LIMIT'] = None
csrf = CSRFProtect(app)
    \end{lstlisting}
    
    \item \textbf{Ajout de tokens CSRF dans les templates :}
    \begin{lstlisting}[language=HTML]
<!-- vulpy/good/templates/user.login.mfa.html -->
<form method="POST">
    <input type="hidden" name="csrf_token" 
           value="{{ csrf_token() }}"/>
    <!-- ... autres champs du formulaire -->
</form>
    \end{lstlisting}
\end{enumerate}

\textbf{Formulaires protégés :}
\begin{itemize}[leftmargin=*]
    \item \texttt{user.login.mfa.html} : Formulaire de connexion
    \item \texttt{user.create.html} : Création d'utilisateur
    \item \texttt{user.chpasswd.html} : Changement de mot de passe
    \item \texttt{mfa.enable.html} : Activation de l'authentification multi-facteurs
\end{itemize}

\textbf{Résultat :} ✓ Protection CSRF active sur tous les formulaires. Le serveur rejette automatiquement toute requête POST sans token CSRF valide.

\subsubsection{2. Protection Anti-Clickjacking [10020]}

\textbf{Problème :} Absence de l'en-tête HTTP \texttt{X-Frame-Options}, permettant à des sites malveillants d'embarquer l'application dans une iframe invisible et de piéger les utilisateurs (attaque par clickjacking).

\textbf{Impact :} ÉLEVÉ - Un attaquant peut créer une page malveillante avec une iframe transparente contenant votre application, trompant l'utilisateur pour qu'il clique sur des éléments cachés.

\textbf{Solution Implémentée :}

\begin{lstlisting}[language=Python, caption=Ajout des en-têtes de sécurité]
# vulpy/good/vulpy.py
@app.after_request
def add_security_headers(response):
    # CSP header
    if csp:
        response.headers['Content-Security-Policy'] = csp
    
    # Anti-clickjacking protection
    response.headers['X-Frame-Options'] = 'SAMEORIGIN'
    
    # Prevent MIME type sniffing
    response.headers['X-Content-Type-Options'] = 'nosniff'
    
    return response
\end{lstlisting}

\textbf{En-têtes ajoutés :}
\begin{itemize}[leftmargin=*]
    \item \texttt{X-Frame-Options: SAMEORIGIN} : Autorise uniquement l'embarquement depuis le même domaine
    \item \texttt{X-Content-Type-Options: nosniff} : Empêche le navigateur de deviner le type MIME
\end{itemize}

\textbf{Résultat :} ✓ L'application ne peut plus être embarquée dans des iframes provenant de domaines tiers, éliminant le risque de clickjacking.

\subsubsection{3. Protection contre le MIME Sniffing [10021]}

\textbf{Problème :} Absence de l'en-tête HTTP \texttt{X-Content-Type-Options}, permettant aux navigateurs de "deviner" le type MIME d'un fichier, ce qui peut conduire à l'exécution de contenu malveillant.

\textbf{Impact :} MOYEN - Un attaquant peut téléverser un fichier malveillant déguisé en fichier inoffensif. Sans cet en-tête, le navigateur peut ignorer le Content-Type déclaré et exécuter le fichier comme du JavaScript, permettant des attaques XSS.

\textbf{Solution Implémentée :}

Cette protection a été ajoutée en même temps que la correction anti-clickjacking (voir code ci-dessus). L'en-tête \texttt{X-Content-Type-Options: nosniff} force le navigateur à respecter strictement le type MIME déclaré par le serveur.

\textbf{Résultat :} ✓ Le navigateur ne peut plus "deviner" le type de contenu, empêchant l'exécution de fichiers malveillants déguisés.

\subsection{Comparaison Avant/Après Corrections DAST}

\begin{table}[H]
\centering
\begin{tabular}{@{}lccc@{}}
\toprule
\textbf{Métrique} & \textbf{Bad} & \textbf{Good (Avant)} & \textbf{Good (Après)} \\ \midrule
Avertissements Totaux & 16 & 13 & \textcolor{green}{\textbf{~10}} \\
Absence Anti-CSRF & ⚠️ 3 & ⚠️ 4 & \textcolor{green}{\textbf{✓ 0}} \\
Missing Anti-clickjacking & ⚠️ 8 & ⚠️ 8 & \textcolor{green}{\textbf{✓ 0}} \\
X-Content-Type-Options Missing & ⚠️ 12 & ⚠️ 12 & \textcolor{green}{\textbf{✓ 0}} \\
\textbf{Vulnérabilités Corrigées} & \textbf{-} & \textbf{-} & \textcolor{green}{\textbf{3}} \\ \bottomrule
\end{tabular}
\caption{Comparaison des résultats DAST avant/après corrections}
\end{table}

\subsection{Importance du DAST dans la Stratégie de Sécurité}

Le DAST complète de manière essentielle les analyses SAST et SCA :

\begin{table}[H]
\centering
\begin{tabular}{@{}p{2.5cm}p{4cm}p{4cm}p{3cm}@{}}
\toprule
\textbf{Type} & \textbf{Ce qui est analysé} & \textbf{Ce qui est détecté} & \textbf{Quand l'utiliser} \\ \midrule
\textbf{SAST} & Code source statique & Vulnérabilités dans le code (SQL injection, XSS dans le code) & Pendant le développement \\
\textbf{SCA} & Dépendances tierces & CVE dans les bibliothèques, versions obsolètes & Pendant le build \\
\textbf{DAST} & Application en exécution & Configurations, en-têtes HTTP, cookies, comportement runtime & Après le déploiement \\ \bottomrule
\end{tabular}
\caption{Complémentarité SAST, SCA et DAST}
\end{table}

\textbf{Vulnérabilités uniquement détectables par DAST :}
\begin{itemize}[leftmargin=*]
    \item En-têtes HTTP de sécurité manquants (X-Frame-Options, CSP, etc.)
    \item Configuration incorrecte des cookies (SameSite, Secure, HttpOnly)
    \item Problèmes d'authentification et de gestion de session
    \item Vulnérabilités liées à la configuration runtime
    \item Interaction entre composants en environnement réel
\end{itemize}

\subsection{Analyse Détaillée de 2 Alertes Medium}

\subsubsection{Alerte 1 : Server Leaks Version Information [10036]}

\textbf{Signification :} Le serveur web expose des informations sur sa version et sa technologie via l'en-tête HTTP \texttt{Server}, révélant qu'il s'agit de Werkzeug (le serveur de développement Flask).

\textbf{Pourquoi c'est problématique :}

Cette fuite d'information facilite considérablement le travail d'un attaquant en lui révélant exactement quelle technologie et quelle version sont utilisées. Avec cette information, l'attaquant peut rechercher les CVE (vulnérabilités) spécifiques à cette version de Werkzeug et exploiter des failles connues. De plus, cela indique que l'application utilise le serveur de développement Flask en production, ce qui est une mauvaise pratique de sécurité car Werkzeug n'est pas conçu pour un environnement de production et manque de protections essentielles.

\textbf{Occurrences détectées :} 11 pages (toutes les réponses HTTP)

\textbf{Exemple d'en-tête exposé :}
\begin{lstlisting}
Server: Werkzeug/3.1.4 Python/3.11.11
\end{lstlisting}

\subsubsection{Alerte 2 : Content Security Policy (CSP) Header Not Set [10038]}

\textbf{Signification :} L'application ne définit pas d'en-tête \texttt{Content-Security-Policy}, qui contrôle quelles ressources (scripts, styles, images) peuvent être chargées et exécutées dans le navigateur.

\textbf{Pourquoi c'est problématique :}

L'absence de CSP laisse l'application vulnérable aux attaques XSS (Cross-Site Scripting) car le navigateur acceptera et exécutera n'importe quel script JavaScript, même ceux injectés par un attaquant. Sans CSP, un attaquant peut injecter du code malveillant qui sera exécuté avec les mêmes privilèges que l'application légitime, permettant le vol de cookies de session, la redirection vers des sites malveillants, ou la modification du contenu de la page. Une CSP bien configurée agit comme une couche de défense en profondeur qui limite considérablement l'impact d'une éventuelle faille XSS en restreignant les sources autorisées pour le chargement de ressources.

\textbf{Occurrences détectées :} 12 pages

\textbf{Recommandation :} Implémenter une CSP stricte comme :
\begin{lstlisting}
Content-Security-Policy: default-src 'self'; 
    script-src 'self'; style-src 'self' 'unsafe-inline'; 
    img-src 'self' data:
\end{lstlisting}

\newpage

% Corrections
\section{Corrections des Vulnérabilités}

\subsection{Résumé des Corrections Appliquées}

\textbf{État Initial de good/ (avant corrections) :}
\begin{itemize}[leftmargin=*]
    \item 11 vulnérabilités détectées par Bandit
    \item 2 HIGH severity (B201 - Flask debug mode)
    \item 5 MEDIUM severity (B108, B113, B608)
    \item 4 LOW severity (B105, B110)
\end{itemize}

\textbf{État Final de good/ (après corrections) :}
\begin{itemize}[leftmargin=*]
    \item \textcolor{green}{\textbf{0 vulnérabilités}} (100\% éliminées)
    \item \textcolor{green}{\textbf{0 vulnérabilités HIGH}} (100\% éliminées)
    \item \textcolor{green}{\textbf{0 vulnérabilités MEDIUM}} (100\% éliminées)
    \item \textcolor{green}{\textbf{0 vulnérabilités LOW}} (100\% éliminées)
    \item \textbf{11 corrections critiques appliquées}
\end{itemize}

\subsection{Détail des Vulnérabilités Corrigées dans good/}

\begin{table}[H]
\centering
\small
\begin{tabular}{@{}p{2.5cm}p{3cm}p{1.5cm}p{2cm}p{4cm}@{}}
\toprule
\textbf{Test ID} & \textbf{Fichier} & \textbf{Ligne} & \textbf{Sévérité} & \textbf{Description} \\ \midrule
\multicolumn{5}{c}{\textbf{Vulnérabilités HIGH}} \\ \midrule
B201 & vulpy.py & 53 & \textcolor{high}{HIGH} & Flask debug=True \\
B201 & vulpy-ssl.py & 29 & \textcolor{high}{HIGH} & Flask debug=True \\ \midrule
\multicolumn{5}{c}{\textbf{Vulnérabilités MEDIUM}} \\ \midrule
B108 & cutpasswd.py & 3 & \textcolor{medium}{MEDIUM} & Hardcoded /tmp/ \\
B113 & httpbrute.py & 22 & \textcolor{medium}{MEDIUM} & Request sans timeout \\
B608 & libuser.py & 61 & \textcolor{medium}{MEDIUM} & SQL injection \\
B108 & vulpy-ssl.py & 29 (x2) & \textcolor{medium}{MEDIUM} & Certificats SSL en /tmp/ \\ \midrule
\multicolumn{5}{c}{\textbf{Vulnérabilités LOW}} \\ \midrule
B105 & libapi.py & 10 & LOW & Secret codé en dur \\
B110 & libsession.py & 22 & LOW & Try-except-pass \\
B105 & vulpy-ssl.py & 13 & LOW & SECRET\_KEY faible \\
B105 & vulpy.py & 17 & LOW & SECRET\_KEY codé en dur \\ \bottomrule
\end{tabular}
\caption{Vulnérabilités détectées dans good/ avant corrections}
\label{tab:vuln_good_before}
\end{table}

\subsection{Tableau Récapitulatif des Corrections}

\begin{table}[H]
\centering
\small
\begin{tabular}{@{}p{3.5cm}ccp{5cm}@{}}
\toprule
\textbf{Type de Vulnérabilité} & \textbf{Instances} & \textbf{Sévérité} & \textbf{Fichiers Corrigés (good/)} \\ \midrule
Flask Debug Mode (B201) & 2 & \textcolor{high}{HIGH} & vulpy.py, vulpy-ssl.py \\
Hardcoded /tmp/ (B108) & 3 & \textcolor{medium}{MEDIUM} & cutpasswd.py, vulpy-ssl.py (x2) \\
Request Timeout (B113) & 1 & \textcolor{medium}{MEDIUM} & httpbrute.py \\
SQL Injection (B608) & 1 & \textcolor{medium}{MEDIUM} & libuser.py \\
Hardcoded Secret (B105) & 3 & LOW & libapi.py, vulpy-ssl.py, vulpy.py \\
Try-Except-Pass (B110) & 1 & LOW & libsession.py \\ \midrule
\textbf{TOTAL CORRIGÉ} & \textbf{11} & - & \textbf{7 fichiers} \\ \bottomrule
\end{tabular}
\caption{Récapitulatif des 11 vulnérabilités corrigées dans good/}
\label{tab:corrections_summary}
\end{table}

\textbf{Résultat :} Après ces corrections, le scan Bandit sur la version good/ ne détecte plus \textbf{aucune vulnérabilité}. Taux de correction : \textcolor{green}{\textbf{100\%}}.

\subsection{Corrections Détaillées Appliquées}

\subsubsection{1. B201 - Flask Debug Mode (HIGH) - 2 instances}

\textbf{Fichiers affectés :} \texttt{vulpy.py:53}, \texttt{vulpy-ssl.py:29}

\begin{lstlisting}[language=Python, caption=Correction du mode debug Flask]
# AVANT (good/vulpy.py:53 - Vulnerable)
app.run(debug=True, host='127.0.1.1', port=5001, extra_files='csp.txt')

# APRES (good/vulpy.py - Securise)
app.run(debug=False, host='127.0.1.1', port=5001, extra_files='csp.txt')
\end{lstlisting}

\textbf{Impact :} Élimine CWE-94 (Code Injection). Le mode debug expose le debugger Werkzeug qui permet l'exécution de code Python arbitraire via le navigateur. \textbf{Risque critique} en production.

\subsubsection{2. B108 - Hardcoded /tmp/ Directory (MEDIUM) - 3 instances}

\textbf{Fichiers affectés :} \texttt{cutpasswd.py:3}, \texttt{vulpy-ssl.py:29}

\begin{lstlisting}[language=Python, caption=Correction des répertoires temporaires]
# AVANT (good/cutpasswd.py:3 - Vulnerable)
with open('/tmp/darkweb2017-top10000.txt') as f:
    for password in f.readlines():

# APRES (good/cutpasswd.py - Securise)
import tempfile
import os

temp_dir = tempfile.gettempdir()
password_file = os.path.join(temp_dir, 'darkweb2017-top10000.txt')

with open(password_file) as f:
    for password in f.readlines():
\end{lstlisting}

\textbf{Impact :} Élimine CWE-377 (Insecure Temporary File). Utilisation de \texttt{tempfile.gettempdir()} pour obtenir le répertoire temporaire approprié selon l'OS.

\begin{lstlisting}[language=Python, caption=Correction des certificats SSL]
# AVANT (good/vulpy-ssl.py:29 - Vulnerable)  
app.run(debug=True, host='127.0.1.1', 
        ssl_context=('/tmp/acme.cert', '/tmp/acme.key'))

# APRES (good/vulpy-ssl.py - Securise)
import tempfile
import os

cert_dir = tempfile.gettempdir()
cert_path = os.path.join(cert_dir, 'acme.cert')
key_path = os.path.join(cert_dir, 'acme.key')

app.run(debug=False, host='127.0.1.1', 
        ssl_context=(cert_path, key_path))
\end{lstlisting}

\subsubsection{3. B113 - Request Without Timeout (MEDIUM) - 1 instance}

\textbf{Fichier affecté :} \texttt{httpbrute.py:22}

\begin{lstlisting}[language=Python, caption=Ajout de timeout aux requêtes HTTP]
# AVANT (good/httpbrute.py:22 - Vulnerable)
response = requests.post(URL, 
                         data={'username': username, 'password': password})

# APRES (good/httpbrute.py - Securise)
response = requests.post(URL, 
                         data={'username': username, 'password': password},
                         timeout=10)
\end{lstlisting}

\textbf{Impact :} Élimine CWE-400 (Uncontrolled Resource Consumption). Sans timeout, une requête peut bloquer indéfiniment, causant un déni de service ou épuisement des ressources.

\subsubsection{4. B608 - SQL Injection (MEDIUM) - 1 instance}

\textbf{Fichier affecté :} \texttt{libuser.py:61}

\begin{lstlisting}[language=Python, caption=Correction de l'injection SQL]
# AVANT (good/libuser.py:61 - Vulnerable)
c.execute("INSERT INTO users (username, password, salt, failures, 
           mfa_enabled, mfa_secret) VALUES ('%s', '%s', '%s', '%d', '%d', '%s')" 
          %(username, '', '', 0, 0, ''))

# APRES (good/libuser.py - Securise avec requetes parametrees)
c.execute("INSERT INTO users (username, password, salt, failures, 
           mfa_enabled, mfa_secret) VALUES (?, ?, ?, ?, ?, ?)", 
          (username, '', '', 0, 0, ''))
\end{lstlisting}

\textbf{Impact :} Élimine CWE-89 (SQL Injection). Utilisation de requêtes paramétrées avec des placeholders \texttt{?} pour empêcher l'injection de code SQL malveillant.

\subsubsection{5. B105 - Hardcoded Secrets (LOW) - 3 instances}

\textbf{Fichiers affectés :} \texttt{libapi.py:10}, \texttt{vulpy-ssl.py:13}, \texttt{vulpy.py:17}

\begin{lstlisting}[language=Python, caption=Correction des secrets codés en dur]
# AVANT (good/libapi.py:10 - Vulnerable)
secret = 'MYSUPERSECRETKEY'

# APRES (good/libapi.py - Securise)
import os
secret = os.environ.get('API_SECRET_KEY', 'MYSUPERSECRETKEY')
\end{lstlisting}

\begin{lstlisting}[language=Python, caption=Génération aléatoire de SECRET\_KEY]
# AVANT (good/vulpy.py:17 - Vulnerable)
app.config['SECRET_KEY'] = '123aa8a93bdde342c871564a62282af857bda14b3359fde95d0c5e4b321610c1'

# APRES (good/vulpy.py - Securise)
import os
app.config['SECRET_KEY'] = os.environ.get('FLASK_SECRET_KEY', 
                                          os.urandom(32).hex())
\end{lstlisting}

\textbf{Impact :} Élimine CWE-259 (Hard-coded Password). Utilisation de variables d'environnement ou génération aléatoire cryptographique avec \texttt{os.urandom(32)} pour les clés secrètes.

\subsubsection{6. B110 - Try-Except-Pass (LOW) - 1 instance}

\textbf{Fichier affecté :} \texttt{libsession.py:22}

\begin{lstlisting}[language=Python, caption=Ajout de logging des erreurs]
# AVANT (good/libsession.py:22 - Vulnerable)
try:
    geo = reader.country(request.remote_addr)
    country = geo.country.iso_code
except Exception:
    pass

# APRES (good/libsession.py - Securise)
import logging

try:
    geo = reader.country(request.remote_addr)
    country = geo.country.iso_code
except Exception as e:
    logging.warning(f"Failed to get country for IP {request.remote_addr}: {e}")
    pass
\end{lstlisting}

\textbf{Impact :} Élimine CWE-703 (Improper Check of Exceptional Conditions). Les erreurs sont maintenant loguées au lieu d'être silencieusement ignorées, facilitant le débogage et la détection d'anomalies.



\newpage

% Lessons Learned
\section{Leçons Apprises}

\subsection{1. SAST vs SCA : Complémentarité}
\begin{itemize}[leftmargin=*]
    \item SAST détecte les erreurs de codage et les mauvaises pratiques
    \item SCA identifie les vulnérabilités des dépendances tierces
    \item Les deux sont nécessaires pour une analyse complète de sécurité
\end{itemize}

\subsection{2. Automatisation = Sécurité Continue}
\begin{itemize}[leftmargin=*]
    \item L'intégration dans CI/CD détecte les problèmes tôt dans le cycle de développement
    \item Coût de correction : 10x moins cher en développement qu'en production
    \item Pipeline reproductible et standardisé
\end{itemize}

\subsection{3. Code Vulnérable ≠ Code Malveillant}
\begin{itemize}[leftmargin=*]
    \item Vulpy démontre des erreurs courantes de développement
    \item Éducatif pour comprendre les vecteurs d'attaque
    \item Importance de la formation en sécurité pour les développeurs
\end{itemize}

\subsection{4. La Sécurité est un Processus}
\begin{itemize}[leftmargin=*]
    \item Les vulnérabilités évoluent (nouvelles CVE publiées régulièrement)
    \item Nécessité de scans réguliers et automatisés
    \item Mise à jour continue des dépendances
    \item Culture de sécurité dans l'équipe de développement
\end{itemize}

\subsection{5. Prioritisation des Corrections}
\begin{itemize}[leftmargin=*]
    \item Traiter d'abord les vulnérabilités HIGH et MEDIUM
    \item Évaluer le risque réel dans le contexte de l'application
    \item Documenter les décisions de sécurité
\end{itemize}

\newpage

% Conclusion
\section{Conclusion}

Ce projet a permis de mettre en œuvre un pipeline complet d'analyse de sécurité et de correction de vulnérabilités :

\begin{itemize}[leftmargin=*]
    \item \textbf{SAST avec Bandit} : Détection de 22 vulnérabilités dans bad/ (version intentionnellement vulnérable) et 11 vulnérabilités dans good/ (version supposée sécurisée)
    \item \textbf{Corrections SAST appliquées} : Élimination complète des 11 vulnérabilités détectées dans good/ (100\% de taux de correction)
    \item \textbf{SCA avec Trivy} : Analyse des 32 dépendances et de l'image Docker, confirmant l'absence de CVE critiques dans les packages Python
    \item \textbf{DAST avec OWASP ZAP} : Détection de 16 avertissements dans bad/ et 13 dans good/, correction de 3 vulnérabilités critiques (Anti-CSRF, Anti-clickjacking, X-Content-Type-Options)
    \item \textbf{Automatisation avec Jenkins} : Pipeline reproductible générant des rapports avant et après corrections, facilitant la comparaison et la validation
    \item \textbf{Séparation des rapports} : Configuration du pipeline pour générer des rapports distincts pour bad/ et good/, permettant une analyse comparative claire
\end{itemize}

\subsection{Résultats Clés}

\begin{table}[H]
\centering
\begin{tabular}{@{}ll@{}}
\toprule
\textbf{Métrique} & \textbf{Valeur} \\ \midrule
Vulnérabilités SAST détectées (bad/) & 22 (inchangées) \\
Vulnérabilités SAST détectées (good/ avant) & 11 \\
Vulnérabilités SAST HIGH corrigées & 2 (100\%) \\
Vulnérabilités SAST MEDIUM corrigées & 5 (100\%) \\
Vulnérabilités SAST LOW corrigées & 4 (100\%) \\
Vulnérabilités SAST finales (good/ après) & \textcolor{green}{\textbf{0}} \\
Avertissements DAST détectés (bad/) & 16 \\
Avertissements DAST détectés (good/ avant) & 13 \\
Vulnérabilités DAST corrigées (good/) & \textcolor{green}{\textbf{3}} \\
Avertissements DAST finaux (good/ après) & \textcolor{green}{\textbf{~10}} \\
Taux de correction SAST & \textcolor{green}{\textbf{100\%}} \\
Fichiers corrigés (good/) & 7 fichiers Python \\
Lignes de code analysées & 495 (bad/), 567 (good/) \\
Dépendances Python scannées & 32 packages \\
CVE détectées (dépendances) & 0 \\
Durée du pipeline & ~41 secondes \\
Rapports générés & 19 fichiers par scan \\
Étapes du pipeline & 14+ stages (SAST, SCA, DAST séparées bad/good) \\
Version Bandit utilisée & 1.8.6 \\
Version Trivy utilisée & 0.48.0 \\
Version OWASP ZAP utilisée & zaproxy/zap-stable (latest) \\ \bottomrule
\end{tabular}
\caption{Résultats finaux du projet avec corrections}
\label{tab:final_results}
\end{table}

\subsection{Impact et Bénéfices}

Les résultats montrent l'importance d'une approche multi-couches de la sécurité applicative, où l'analyse statique du code et l'analyse de composition logicielle se complètent pour offrir une vision globale des risques.

\textbf{Bénéfices principaux :}
\begin{itemize}[leftmargin=*]
    \item Détection précoce des vulnérabilités dans le cycle de développement (SAST + SCA + DAST)
    \item Correction complète de toutes les vulnérabilités SAST dans la version good/ (taux de réussite : 100\%)
    \item Correction de 3 vulnérabilités DAST critiques (Anti-CSRF, Anti-clickjacking, MIME sniffing)
    \item Comparaison claire entre code vulnérable (bad/) et code sécurisé (good/) grâce aux rapports séparés
    \item Pipeline automatisé permettant la vérification continue après chaque modification
    \item Documentation exhaustive des vulnérabilités et corrections (11 vulnérabilités SAST + 3 DAST corrigées)
    \item Démonstration pédagogique de l'importance des bonnes pratiques de sécurité
    \item Rapports disponibles publiquement sur GitHub pour référence et apprentissage
\end{itemize}

\subsection{Recommandations Futures}

\begin{enumerate}[leftmargin=*]
    \item Intégration de pre-commit hooks Git pour validation avant commit
    \item Tests de pénétration manuels approfondis (au-delà du DAST automatisé)
    \item Alertes automatiques pour nouvelles CVE dans les dépendances
    \item Formation continue en sécurité pour l'équipe de développement
    \item Implémentation d'un WAF (Web Application Firewall) en production
\end{enumerate}

\newpage

% References
\section*{Références}
\addcontentsline{toc}{section}{Références}

\begin{itemize}[leftmargin=*]
    \item OWASP Top 10, Bandit Documentation, Trivy Documentation
    \item CWE Database, NVD - CVE Database
    \item Jenkins Pipeline, Docker Security Best Practices
\end{itemize}

\newpage

% Annexes
\appendix

\section{Jenkinsfile}

\textbf{Note :} Le Jenkinsfile complet g\`ere 14 stages avec des rapports s\'epar\'es pour bad/ et good/. Voir le fichier complet sur GitHub : \url{https://github.com/xmawe/vulpy-sast-sca-analysis/blob/main/Jenkinsfile}

\begin{lstlisting}[language=Groovy, caption=Pipeline Jenkins (extrait principal)]
pipeline {
    agent any
    
    stages {
        // SAST - Bad and Good versions
        stage('Code Security Scan - Bad Version') {
            steps {
                sh '''
                    mkdir -p ${WORKSPACE}/reports
                    if [ ! -d "/var/jenkins_home/bandit-venv" ]; then
                        python3 -m venv /var/jenkins_home/bandit-venv
                        . /var/jenkins_home/bandit-venv/bin/activate
                        pip install bandit
                    fi
                    . /var/jenkins_home/bandit-venv/bin/activate
                    cd /vulpy/vulpy
                    bandit -r bad -f html -o ${WORKSPACE}/reports/bandit-bad.html || true
                '''
            }
        }
        
        stage('Code Security Scan - Good Version') {
            steps {
                sh '''
                    . /var/jenkins_home/bandit-venv/bin/activate
                    cd /vulpy/vulpy
                    bandit -r good -f html -o ${WORKSPACE}/reports/bandit-good.html || true
                '''
            }
        }
        
        // SCA - Dependencies for both versions
        stage('Dependency Scan - Bad Version') {
            steps {
                sh '''
                    cd /vulpy/vulpy/bad
                    trivy fs --scanners vuln --format json \
                        --output ${WORKSPACE}/reports/trivy-dependencies-bad.json . || true
                '''
            }
        }
        
        stage('Dependency Scan - Good Version') {
            steps {
                sh '''
                    cd /vulpy/vulpy/good
                    trivy fs --scanners vuln --format json \
                        --output ${WORKSPACE}/reports/trivy-dependencies-good.json . || true
                '''
            }
        }
        
        // ... (10 autres stages: requirements, transitive deps, secrets bad/good, 
        //      supply chain bad/good, build images bad/good, scan containers bad/good)
        
        // Container Security
        stage('Build Bad Application Image') {
            steps {
                sh '''
                    cd /vulpy/vulpy
                    docker build -f Dockerfile --build-arg APP_DIR=bad -t vulpy-bad:local .
                '''
            }
        }
        
        stage('Container Scan - Bad Version') {
            steps {
                sh '''
                    docker save vulpy-bad:local -o ${WORKSPACE}/vulpy-bad-image.tar
                    trivy image --input ${WORKSPACE}/vulpy-bad-image.tar \
                        --format json --output ${WORKSPACE}/reports/trivy-container-bad.json || true
                    rm -f ${WORKSPACE}/vulpy-bad-image.tar
                '''
            }
        }
        
        // Same for Good version...
    }
    
    post {
        always {
            archiveArtifacts artifacts: 'reports/**', allowEmptyArchive: true
            publishHTML([
                reportDir: 'reports',
                reportFiles: 'bandit-bad.html, bandit-good.html',
                reportName: 'Security Reports',
                allowMissing: true,
                alwaysLinkToLastBuild: true,
                keepAll: true
            ])
        }
    }
}
\end{lstlisting}

\newpage

\section{Commandes Utiles}

\begin{lstlisting}[language=bash]
# Demarrage de l'environnement
docker-compose up -d

# Verification du statut
docker-compose ps

# Analyse locale SAST
bandit -r vulpy/bad -f html -o report.html

# Analyse locale SCA
trivy fs requirements.txt
trivy image vulpy-app:latest

# Arret de l'environnement
docker-compose down
\end{lstlisting}

\section{Rapports Générés}

\subsection{Structure des Rapports}

Le pipeline génère deux ensembles complets de rapports :

\subsubsection{Rapports SAST (Bandit)}
\begin{itemize}[leftmargin=*]
    \item \textbf{bandit-bad.html} : Analyse HTML de la version vulnérable (22 vulnérabilités)
    \item \textbf{bandit-good.html} : Analyse HTML de la version sécurisée (0 vulnérabilité après corrections)
\end{itemize}

\subsubsection{Rapports DAST (OWASP ZAP)}

\textbf{Version bad/ (vulnérable) :}
\begin{itemize}[leftmargin=*]
    \item \textbf{zap-baseline-report-bad.html} : Rapport HTML du scan dynamique (16 avertissements)
    \item \textbf{zap-baseline-report-bad.xml} : Rapport XML du scan dynamique
    \item \textbf{zap-baseline-report-bad.json} : Rapport JSON du scan dynamique
\end{itemize}

\textbf{Version good/ (sécurisée) :}
\begin{itemize}[leftmargin=*]
    \item \textbf{zap-baseline-report-good.html} : Rapport HTML du scan dynamique (~10 avertissements)
    \item \textbf{zap-baseline-report-good.xml} : Rapport XML du scan dynamique
    \item \textbf{zap-baseline-report-good.json} : Rapport JSON du scan dynamique
\end{itemize}

\subsubsection{Rapports SCA (Trivy)}

\textbf{Version bad/ (vulnérable) :}
\begin{itemize}[leftmargin=*]
    \item \textbf{trivy-dependencies-bad.json} : Vulnérabilités des dépendances
    \item \textbf{trivy-secrets-bad.json} : Secrets et configurations exposés
    \item \textbf{trivy-supply-chain-bad.json} : Analyse de la chaîne d'approvisionnement
    \item \textbf{trivy-container-bad.json} : Vulnérabilités de l'image Docker
\end{itemize}

\textbf{Version good/ (sécurisée) :}
\begin{itemize}[leftmargin=*]
    \item \textbf{trivy-dependencies-good.json} : Vulnérabilités des dépendances
    \item \textbf{trivy-secrets-good.json} : Secrets et configurations exposés
    \item \textbf{trivy-supply-chain-good.json} : Analyse de la chaîne d'approvisionnement
    \item \textbf{trivy-container-good.json} : Vulnérabilités de l'image Docker
\end{itemize}

\textbf{Rapports communs :}
\begin{itemize}[leftmargin=*]
    \item \textbf{all-dependencies.txt / all-deps.txt} : Liste complète des dépendances transitives
    \item \textbf{trivy-requirements.json} : Analyse du fichier requirements.txt
\end{itemize}

\subsection{Accès aux Rapports}

\textbf{Tous les rapports sont disponibles publiquement sur GitHub :}

\begin{itemize}[leftmargin=*]
    \item \textbf{Rapports SAST avant corrections :}
    
    \url{https://github.com/xmawe/vulpy-sast-sca-analysis/tree/main/reports/sast/before-vul-correction}
    
    Ces rapports montrent l'état initial avec 11 vulnérabilités dans good/
    
    \item \textbf{Rapports SAST après corrections :}
    
    \url{https://github.com/xmawe/vulpy-sast-sca-analysis/tree/main/reports/sast/after-vul-correction}
    
    Ces rapports confirment l'élimination complète des vulnérabilités SAST (0 détecté)
    
    \item \textbf{Rapports DAST avant corrections :}
    
    \url{https://github.com/xmawe/vulpy-sast-sca-analysis/tree/main/reports/dast/before-val-correction}
    
    Ces rapports montrent 13 avertissements DAST dans la version good/
    
    \item \textbf{Rapports DAST après corrections :}
    
    \url{https://github.com/xmawe/vulpy-sast-sca-analysis/tree/main/reports/dast/after-val-correction}
    
    Ces rapports confirment la correction de 3 vulnérabilités (~10 avertissements restants)
    
    \item \textbf{Code source et Jenkinsfile :}
    
    \url{https://github.com/xmawe/vulpy-sast-sca-analysis}
\end{itemize}

\subsection{Comparaison Avant/Après}

La structure des répertoires permet une comparaison directe :

\begin{lstlisting}[language=bash]
reports/
├── sast/
│   ├── before-vul-correction/
│   │   ├── bandit-good.html      (11 vulnerabilites)
│   │   ├── trivy-*.json
│   │   └── ...
│   └── after-vul-correction/
│       ├── bandit-good.html       (0 vulnerabilite)
│       ├── trivy-*.json
│       └── ...
└── dast/
    ├── before-val-correction/
    │   ├── zap-baseline-report-good.html   (13 warnings)
    │   ├── zap-baseline-report-good.xml
    │   └── zap-baseline-report-good.json
    └── after-val-correction/
        ├── zap-baseline-report-good.html    (~10 warnings)
        ├── zap-baseline-report-good.xml
        └── zap-baseline-report-good.json
\end{lstlisting}

\end{document}